\section{Defining the Scope}
A valuable process that will result in a valuable product. To know milestones of the project.
\subsection{Reason \#1: So You Know What Youre Building}
We should have a concrete understanding of the requirements of the product, so that individual people have a common understanding of the scope of the project. With specifying the 
requirements, dependencies are well-understood \_ perhaps something is missing from current scope\_.
\subsection{Reason \#2: So You Know What Youre Not Building}
\section{Functionality and Content}
Why do we make this product? What are we going to make? At this step we determine the content and the functionality that we need
 to consiuder. Functionality and content are interchanably called, \textit{feature} in this booklet.
\section{Defining Requirements}
\begin{itemize}
    \item Branding requirement
    \item Supported browsers
    \item Operating systems
\end{itemize}
\section{Functional Specifications}
They should be light weight
\subsection{Writing It Down}
Specs should be:
\begin{itemize}
    \item \textbf{positive: } do not describe baf thing. describe how to prevent it.
    \item \textbf{specific: } do not leave any space for interpretation.
    \item \textbf{avoid subjective language: } just like being specific. 
\end{itemize}
\section{Content Requirements}
Includes videos, texts and images
\section{Prioritizing Requirements}
Sort features that should be included in the project. Features should fulfill our strategic goals. 
Some features are impossible to implement (to smell products). 